% vim: set tw=78 sts=2 sw=2 ts=8 aw et ai:

The testbed described in the previous section was used to run test
for low and high bandwidths, starting from 1 Mbps and doubling
up to 1000 Mbps, which is a limit imposed by the Mininet framework itself.
In figures \ref{fig:16mbps-cw} and \ref{fig:16mbps-d}  we can
observe the influence of two major parameters on the throughput for an average
value of 16 Mbps for the bandwidth of each link and no loss on either channel.
The setup is the one that was described in the previous section.

\begin{figure}
  \centering
  \includegraphics[width=0.85\textwidth]{img/throughput-cwnd-16Mbps}
  \caption{Initial congestion window impact}
  \label{fig:16mbps-cw}
\end{figure}

\begin{figure}
  \centering
  \includegraphics[width=0.85\textwidth]{img/throughput-delay-16Mbps}
  \caption{Channel delay impact}
  \label{fig:16mbps-d}
\end{figure}

Figure \ref{fig:16mbps-cw} shows the impact of the congestion window's initial
size correlated with the receiving queue size. The channel has no losses and
the link delay is 80 milliseconds. It is noticeable that the initial size of
the congestion window does not affect performance too much because MPTCP's
algorithm is adapting it over the course of the data transfer. Throughput is
better for larger receive buffers, but it reaches a ceiling of approximately
10 Mbps because of the sizeable delay.

Figure \ref{fig:16mbps-d} outlines the impact of delay. MPTCP's performance is
close to the maximum possible throughput of 32 Mbps regardless of the receive
buffer when there is zero delay. The performance expectedly decreases when delay
is larger and we notice that the receive buffer size starts to influence
throughput. That is because the size of the receive buffer is conditioned by the
bandwidth-delay product as it was mentioned in Section \ref{sec:related}.

\begin{figure}
  \centering
  \includegraphics[width=\textwidth]{img/throughput-bdw-avg}
  \caption{Link capacity impact with average delay}
  \label{fig:bdw-avg}
\end{figure}

\begin{figure}
  \centering
  \includegraphics[width=\textwidth]{img/throughput-bdw-max}
  \caption{Link capacity impact with no delay}
  \label{fig:bdw-max}
\end{figure}

\begin{figure}
  \centering
  \includegraphics[width=0.85\textwidth]{img/throughput-cwnd-notc}
  \caption{Throughput without overhead associated with traffic control}
  \label{fig:cwnd-notc}
\end{figure}

