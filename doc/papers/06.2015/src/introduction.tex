% vim: set tw=78 sts=2 sw=2 ts=8 aw et ai:
Previous work on the project focused on identifying the optimal conditions for maximum MPTCP throughput over OpenVPN tunnels. In the first stage, we identified and tested a set of system and network configurations for enhanced performance. Our main finding was that the best MPTCP behavior is achieved when the socket buffer is twice the size of the bandwidth-delay product \cite{sem2}. In the second stage we investigated the effects of different congestion control algorithms over a variety of network setups. Algorithms such as OLIA and wVegas were compared with OLIA found to offer the best throughput \cite{sem3}. All our experiments were run inside a virtualized testbed, at first using the Mininet network simulator and then switching to a KVM environment.

The single use case we had in mind at the start was that of a user who sits behind a firewall with only a few common ports left open, e.g. HTTP, HTTPS, DNS. Opening tunnels on all these ports and joining them under the same MPTCP connection would enable any application to bypass the egress firewall while also maximizing its throughput. As work on the project progressed, more use cases became apparent. A very common situation is that of mobile network carriers enforcing traffic shaping policies targeted at (a) over-the-network phone call applications such as Skype or (b) certain common ports. Furthermore, our tests showed that some mobile network operators use middleboxes that filter MPTCP options altogether, making tunneling mandatory if any kind of multipath communication is to be achieved. Therefore our solution could be used to enable mobile users to take advantage of all of their applications. In addition, virtually all smartphones today are equipped with multiple network interfaces (WiFi, 3/4G) - MPTCP bandwidth aggregation can further improve performance. For these we have decided to build a prototype aimed at mobile devices. We have deployed the prototype on a regular smartphone and tested it over a well-known mobile network carrier.

The next section describes some of the related research on the effort to mitigate middlebox interference on MPTCP. Section \ref{sec:setup} showcases our prototype and details our setup. Section \ref{sec:results} details our experiments and analyzes the results. Conclusions and future research directions are presented in section \ref{sec:conclusion}.